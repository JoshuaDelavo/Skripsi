\chapter{Kesimpulan dan Saran}
\label{chap:kesimpulan}

\section{Kesimpulan}
Berdasarkan hasil dari analisis, implementasi, dan pengujian Aplikasi VisKur yang telah dibuat, telah diperoleh kesimpulan sebagai berikut:

\begin{itemize}
    \item Berdasarkan tujuan dari skripsi ini yang berfungsi untuk membantu mahasiswa informatika Universitas Katolik Parahyangan sudah dapat berhasil terlaksanakan karena dilihat dari hasil survei yang dilakukan sebagian besar mahasiswa lebih memilih menggunakan aplikasi VisKur daripada menggunakan pohon kurikulum untuk melihat kurikulum 2018 dengan alasan yang telah disebutkan pada \ref{pengujianEksperimental}.
    
    \item Aplikasi VisKur telah dapat mengambil data dari \textit{API} kemudian membuatkan visualisasinya dalam bentuk \textit{Network} dan \textit{Timeline}.
    
    \item Aplikasi VisKur telah dapat dipasang dan dijalankan pada seluruh perangkat dengna sistem operasi \textit{windows}, baik \textit{windows} 10 maupun \textit{windows} 11.
\end{itemize}

\section{Saran}
Dari hasil penelitian termasuk kesimpulan yang didapat, berikut adalah saran untuk pengembang selanjutnya:

\begin{itemize}
    \item Memperbaiki tampilan aplikasi VisKur sehingga \textit{button Network} dan \textit{Timeline} tidak sesederhana saat ini.
    
    \item Penambahan fitur saran pengambilan matakuliah untuk mahasiswa untuk mata kuliah di semester berikutnya secara umum.
\end{itemize}

