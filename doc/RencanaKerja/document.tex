\documentclass[a4paper,twoside]{article}
\usepackage[T1]{fontenc}
\usepackage[bahasa]{babel}
\usepackage{graphicx}
\usepackage{graphics}
\usepackage{float}
\usepackage[cm]{fullpage}
\pagestyle{myheadings}
\usepackage{etoolbox}
\usepackage{setspace} 
\usepackage{lipsum}
\usepackage{verbatim}
\usepackage{url}
\usepackage{hyperref}
\setlength{\headsep}{30pt}
\usepackage[inner=2cm,outer=2.5cm,top=2.5cm,bottom=2cm]{geometry} %margin
% \pagestyle{empty}

\makeatletter
\renewcommand{\@maketitle} {\begin{center} {\LARGE \textbf{ \textsc{\@title}} \par} \bigskip {\large \textbf{\textsc{\@author}} }\end{center} }
\renewcommand{\thispagestyle}[1]{}
\markright{\textbf{\textsc{AIF184001 \textemdash Rencana Kerja Skripsi \textemdash Sem. Genap 2021/2022}}}

\newcommand{\HRule}{\rule{\linewidth}{0.4mm}}
\renewcommand{\baselinestretch}{1}
\setlength{\parindent}{0 pt}
\setlength{\parskip}{6 pt}

\onehalfspacing
\hyphenation{}
\begin{document}

\title{\@judultopik}
\author{\nama \textendash \@npm} 

%tulis nama dan NPM anda di sini:
\newcommand{\nama}{Joshua Delavo}
\newcommand{\@npm}{2017730028}
\newcommand{\@judultopik}{Visualisasi Kurikulum 2018 dengan vis.js dan Electron} % Judul/topik anda
\newcommand{\jumpemb}{1} % Jumlah pembimbing, 1 atau 2
\newcommand{\tanggal}{12/03/2021}

% Dokumen hasil template ini harus dicetak bolak-balik !!!!

\maketitle

\pagenumbering{arabic}

\section{Deskripsi}

\textit{Electron} adalah sebuah kerangka kerja untuk membuat aplikasi desktop menggunakan \textit{JavaScript}, \textit{HTML}, dan \textit{CSS}. Aplikasi ini dapat dikemas untuk dapat berjalan langsung di \textit{macOS}, \textit{Windows}, dan \textit{Linux}. Bisa juga didistribusikan melalui \textit{Mac App Store} atau \textit{Microsoft Store}. Biasanya, jika kita ingin membuat aplikasi \textit{desktop} untuk sistem operasi (OS), kita harus menggunakan kerangka kerja aplikasi khusus untuk setiap sistem operasinya. \textit{Electron} memungkinkan kita untuk membuat aplikasi sekali saja dengan menggunakan teknologi yang telah kita ketahui.

\textit{Vis js} adalah sebuah \textit{library} visualisasi berbasis \textit{browser} yang bersifat dinamis. \textit{Library} ini dirancang agar mudah digunakan untuk menangani data dinamis dalam jumlah yang besar dan memungkinkan untuk memanipulasi serta berinteraksi dengan data tersebut. \textit{Library} ini terdiri dari komponen - komponen, seperti \textit{DataSet}, \textit{Timeline}, \textit{Network}, \textit{Graph2d}, dan \textit{Graph3d}.

Saat mahasiswa akan melakukan \textit{FRS}, seringkali mereka kesulitan untuk melihat kurikulum tahun ajaran yang berlaku. Karena setiap kurikulum memiliki aturan yang berbeda dalam pengambilan matakuliah ataupun matakuliah yang disediakan, maka mahasiswa kadang bingung untuk memilih matakuliah apa yang akan diambil di semester berikutnya.

Maka dari itu, pada skripsi ini akan dibuat sebuah aplikasi visualisasi kurikulum 2018 berbasis \textit{Electron} dengan menggunakan \textit{library Vis js}. Dengan aplikasi ini diharapkan mahasiswa dapat lebih mudah untuk melihat kurikulum yang ada sehingga mempermudah mereka untuk meemilih matakuliah apa yang akan diambil di semester berikutnya, kemudian dengan penggunaan \textit{framework cross platform Electron} ini diharapkan semua mahasiswa pengguna \textit{macOs}, \textit{Windows}, dan \textit{Linux} dapat mengaksesnya dengan mudah.

\section{Rumusan Masalah}
Rumusan masalah yang akan dibahas pada skripsi ini adalah : 
\begin{itemize}
    \item Bagaimana cara memvisualisasikan kurikulum 2018 dalam bentuk \textit{tree}?
    \item Bagaimana cara memvisualisasikan kurikulum 2018 dalam bentuk \textit{timeline}?
\end{itemize}


\section{Tujuan}
Tujuan yang akan dicapai dari penulisan skripsi ini adalah : 
\begin{itemize}
    \item Memahami cara memvisualisasikan kurikulum 2018 dalam bentuk \textit{tree}.
    \item Memahami cara memvisualisasikan kurikulum 2018 dalam bentuk \textit{timeline}.
\end{itemize}

\section{Deskripsi Perangkat Lunak}

Perangkat lunak yang akan dibuat memiliki fitur sebagai berikut :
\begin{itemize}
    \item Perangkat lunak dapat menarik data dari \url{https://raw.githubusercontent.com/ftisunpar/data/master/prasyarat.json}
    \item Perangkat lunak dapat memvisualisasikan kurikulum 2018 dalam bentuk \textit{tree}
    \item Perangkat lunak dapat memvisualisasikan kurikulum 2018 dalam bentuk \textit{timeline}
    \item Perangkat lunak dapat berpindah halaman antar visualisasi \textit{tree} dan \textit{timeline}
\end{itemize}

\section{Detail Pengerjaan Skripsi}

Bagian - bagian pengerjaan skripsi ini adalah :
\begin{enumerate}
    \item Melakukan studi tentang \textit{framework Electron} dan \textit{Library Vis js}.
    \item Mempelajari cara membuat aplikasi berbasis \textit{Electron}.
    \item Mempelajari cara memvisualisasikan data dalam bentuk \textit{tree} dan \textit{timeline} dengan \textit{Vis js}.
    \item Mempelajari data kurikulum 2018 di github beserta cara pengambilan datanya.
    \item Merancang aplikasi berbasis \textit{Electron}.
    \item Merancang visualisasi kurikulum 2018 dalam bentuk \textit{tree}.
    \item Merancang visualisasi kurikulum 2018 dalam bentuk \textit{timeline}.
    \item Mendesain antarmuka aplikasi.
    \item Melakukan pengujian dan ekperimen.
    \item Membuat dokumen skripsi.
\end{enumerate}

\section{Rencana Kerja}

Rincian capaian yang direncanakan di Skripsi 1 adalah sebagai berikut:
\begin{enumerate}
    \item Melakukan studi tentang \textit{framework Electron} dan \textit{Library Vis js}.
    \item Mempelajari cara membuat aplikasi berbasis \textit{Electron}.
    \item Mempelajari cara memvisualisasikan data dalam bentuk \textit{tree} dan \textit{timeline} dengan \textit{Vis js}.
    \item Mempelajari data kurikulum 2018 di github beserta cara pengambilan datanya.
    \item Membuat dokumen skripsi untuk bab pendahuluan, landasan teori, dan analisis.
\end{enumerate}

\newpage
Sedangkan yang akan diselesaikan di Skripsi 2 adalah sebagai berikut:
\begin{enumerate}
\item Merancang aplikasi berbasis \textit{Electron}.
    \item Merancang visualisasi kurikulum 2018 dalam bentuk \textit{tree}.
    \item Merancang visualisasi kurikulum 2018 dalam bentuk \textit{timeline}.
    \item Mendesain antarmuka aplikasi.
    \item Melakukan pengujian dan ekperimen.
    \item Menyelesaikan dokumen skripsi.
\end{enumerate}

\vspace{1cm}
\centering Bandung, \tanggal\\
\vspace{2cm} \nama \\ 
\vspace{1cm}

Menyetujui, \\
\ifdefstring{\jumpemb}{2}{
\vspace{1.5cm}
\begin{centering} Menyetujui,\\ \end{centering} \vspace{0.75cm}
\begin{minipage}[b]{0.45\linewidth}
% \centering Bandung, \makebox[0.5cm]{\hrulefill}/\makebox[0.5cm]{\hrulefill}/2013 \\
\vspace{2cm} Nama: \makebox[3cm]{\hrulefill}\\ Pembimbing Utama
\end{minipage} \hspace{0.5cm}
\begin{minipage}[b]{0.45\linewidth}
% \centering Bandung, \makebox[0.5cm]{\hrulefill}/\makebox[0.5cm]{\hrulefill}/2013\\
\vspace{2cm} Nama: \makebox[3cm]{\hrulefill}\\ Pembimbing Pendamping
\end{minipage}
\vspace{0.5cm}
}{
% \centering Bandung, \makebox[0.5cm]{\hrulefill}/\makebox[0.5cm]{\hrulefill}/2013\\
\vspace{2cm} Pascal Alfadian\\ Pembimbing Tunggal
}
\end{document}